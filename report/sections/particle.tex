\subsection{Particle}
We used a particle fluid simulation method developed by Monaghan \cite{Monaghan1992SPH}, called Smoothed Particle Hydrodynamics (SPH).
Our code focused on modeling using SPH formulations with fluid forces such as pressure and viscosity.
SPH is an interpolation method that evaluates field quantities of each particle based on its local neighborhood using radial symmetrical smoothing kernels.

\subsubsection{Algorithm}
The core steps of the SPH particle fluid simulation is summarized in Algorithm~\ref{alg:sph}.

\begin{algorithm}[h]
\caption{SPH Particle Update Loop}\label{alg:sph}
\begin{algorithmic}[1]
    \State //\ Compute density and pressure
    \For{each particle $i$}
        \For{each neighboring particle $j$}
            \State Compute distance
            \If{Within kernel radius}
                \State Add density contribution $\rho_i = m_j \cdot W(r_{ij}, h)$
            \EndIf
        \EndFor
        \State Compute pressure from density $P = k_p(\rho - \rho_0)$
    \EndFor

    \State //\ Compute forces on each particle
    \For{each particle $i$}
        \State Initialize $f_p \gets 0$, $f_v \gets 0$
        \For{each neighboring particle $j$}
            \State Compute distance
            \If{Within kernel radius}
                \State Compute pressure force contribution
                \State $f_p = m_j \cdot \frac{p_i + p_j}{2\rho_j} \cdot \nabla W(r_ij, h)$
                \State Compute viscosity force contribution
                \State $f_p = m_j \cdot \frac{v_j - v_i}{2\rho_j} \cdot \nabla^2 W(r_ij, h)$
            \EndIf
        \EndFor
        \State Compute gravity force contribution $f_g = G \cdot \frac{m_i}{\rho_i}$
        \State Total force on particle $f_i = f_p + f_v + f_g$
    \EndFor

    \State //\ Integrate velocity and update positions
    \For{each particle $i$}
        \State Update velocity $v_i^{t+\Delta t} = v_i^t = \Delta t \cdot \frac{f_i^t}{\rho_i}$
        \State Update position ${x}_i^{t+\Delta t} = {x}_i^t + \Delta t \cdot {v}_i^{t+\Delta t}$
        \If{position $x_i$ hits domain boundary}
            \State Dampen velocity
            \State Clamp position to boundary
        \EndIf
    \EndFor
\end{algorithmic}
\end{algorithm}

\subsubsection{Intermediate Results and Diagrams}
We initialized our SPH simulation with a slight jitter to each particle's initial positions. This is because SPH evalautes its fields based on its neighbors,
and if particles left and right are evenly spaced, the horizontal forces will cancel out perfectly. This would lead to the particles only bouncing vertically.
