\section{Methods}
To accomplish our project goals, we implemented five distinct 2D incompressible fluid simulation methods—Stable Fluids, SPH, PIC, PIC/FLIP, and APIC—using C++ for the core simulation and OpenGL with GLUT for real-time visualization. Each method was developed independently based on its underlying physical principles and algorithmic structure. We focused on observing and comparing the visual behavior and numerical characteristics of each simulation through qualitative analysis. The following sections describe the implementation details and key observations for each method.

\textbf{Tools and Learning} We used C++ for simulation logic and OpenGL with GLUT for real-time visualization across all simulation methods. The Eigen library was employed for efficient matrix operations, particularly for APIC and FLIP methods where affine velocity matrices were involved. 

Throughout the project, we learned how to structure particle-grid transfer systems, implement spatial neighborhood queries using a sorting grid, and visualize thousands of particles in real time. We also gained practical experience with parallel programming, numerical debugging, and enforcing boundary conditions on staggered MAC grids.

\textbf{Course Content Reference} We applied key concepts from the course, including hybrid fluid simulation methods (PIC, FLIP, APIC), particle-grid transfers, SPH kernel functions, external forces, and pressure projection. These topics directly guided our simulation and implementation strategy.

% \begin{figure}[!htpb]
%     \centering
%     \includegraphics[width=\linewidth]{Figures/PezizaTuberosa.jpg}
%     \caption{Illustration of the fungus Dumontinia tuberosa by physician, mycologist, and illustrator Charles Tulasne (1816–1884) in the book Selecta Fungorum Carpologia (1861–65). (Name of the original work: Peziza tuberosa parasite on Anemone nemorosa).}
%     \label{fig:tcanther}
% \end{figure}

% \subsection{Grid}

The first method implemented in our project is the Stable Fluids method introduced by Stam  \citep{stam2023stable}. This grid-based Eulerian approach uses a fixed discretized grid to represent fluid properties, such as velocity and density fields. The method ensures unconditional stability at the expense of numerical diffusion, making it robust for real-time applications.

The numerical solver follows four main computational steps:

\begin{enumerate}
\item \textbf{Add Source}: Introduce external quantities (density, velocity) into the simulation. Each cell's value is incremented by a source term scaled by the simulation timestep:
\[
x_{i,j} \leftarrow x_{i,j} + \Delta t \cdot s_{i,j}
\]

\item \textbf{Diffuse}: Account for viscosity by spreading fluid properties across the grid. This step employs iterative Gauss-Seidel or Jacobi methods to solve the diffusion equation implicitly:
\[
\frac{x_{i,j}^{t+1} - x_{i,j}^{t}}{\Delta t} = \nu \nabla^2 x_{i,j}^{t+1}
\]

\item \textbf{Project}: Enforce incompressibility by adjusting the velocity field to be divergence-free. The divergence is computed, a pressure field is solved via iterative Jacobi relaxation, and then the pressure gradient is subtracted from the velocity:
\[
\nabla^2 p = \nabla \cdot \mathbf{u}, \quad \mathbf{u} \leftarrow \mathbf{u} - \nabla p
\]

\item \textbf{Advect}: Transport fluid properties through the velocity field. Each grid cell is traced backward in time along the velocity field, and bilinear interpolation reconstructs values:
\[
x_{i,j}^{t+1} = x(\mathbf{p} - \Delta t \cdot \mathbf{u}(\mathbf{p},t), t)
\]
\end{enumerate}

Finally, appropriate \textbf{Boundary Conditions} are applied after each step: velocity components are inverted at solid boundaries, and scalar fields maintain values by copying adjacent interior cells. These operations are encapsulated within the functions \texttt{vel\_step()} and \texttt{dens\_step()}, which are sequentially called in the main simulation loop (\texttt{simulation()}).

\subsection{Particle}
We used a particle fluid simulation method developed by Monaghan \cite{Monaghan1992SPH}, called Smoothed Particle Hydrodynamics (SPH).
Our code focused on modeling using SPH formulations with fluid forces such as pressure and viscosity.
SPH is an interpolation method that evaluates field quantities of each particle based on its local neighborhood using radial symmetrical smoothing kernels.

\subsubsection{Algorithm}
The core steps of the SPH particle fluid simulation is summarized in Algorithm~\ref{alg:sph}.

\begin{algorithm}[h]
\caption{SPH Particle Update Loop}\label{alg:sph}
\begin{algorithmic}[1]
    \State //\ Compute density and pressure
    \For{each particle $i$}
        \For{each neighboring particle $j$}
            \State Compute distance
            \If{Within kernel radius}
                \State Add density contribution $\rho_i = m_j \cdot W(r_{ij}, h)$
            \EndIf
        \EndFor
        \State Compute pressure from density $P = k_p(\rho - \rho_0)$
    \EndFor

    \State //\ Compute forces on each particle
    \For{each particle $i$}
        \State Initialize $f_p \gets 0$, $f_v \gets 0$
        \For{each neighboring particle $j$}
            \State Compute distance
            \If{Within kernel radius}
                \State Compute pressure force contribution
                \State $f_p = m_j \cdot \frac{p_i + p_j}{2\rho_j} \cdot \nabla W(r_ij, h)$
                \State Compute viscosity force contribution
                \State $f_p = m_j \cdot \frac{v_j - v_i}{2\rho_j} \cdot \nabla^2 W(r_ij, h)$
            \EndIf
        \EndFor
        \State Compute gravity force contribution $f_g = G \cdot \frac{m_i}{\rho_i}$
        \State Total force on particle $f_i = f_p + f_v + f_g$
    \EndFor

    \State //\ Integrate velocity and update positions
    \For{each particle $i$}
        \State Update velocity $v_i^{t+\Delta t} = v_i^t = \Delta t \cdot \frac{f_i^t}{\rho_i}$
        \State Update position ${x}_i^{t+\Delta t} = {x}_i^t + \Delta t \cdot {v}_i^{t+\Delta t}$
        \If{position $x_i$ hits domain boundary}
            \State Dampen velocity
            \State Clamp position to boundary
        \EndIf
    \EndFor
\end{algorithmic}
\end{algorithm}

\subsubsection{Intermediate Results and Diagrams}
We initialized our SPH simulation with a slight jitter to each particle's initial positions. This is because SPH evalautes its fields based on its neighbors,
and if particles left and right are evenly spaced, the horizontal forces will cancel out perfectly. This would lead to the particles only bouncing vertically.

% \subsection{PIC}

The Particle-In-Cell (PIC) method uses both particles and grids for fluid simulation. Each simulation step involves five primary phases:

\begin{enumerate}
\item \textbf{Transfer to Grid}: Particle velocities are transferred to nearby grid cells using B-spline weighting functions.

\item \textbf{Apply Gravity}: Gravity force is applied directly to vertical grid velocities.

\item \textbf{Solve Pressure}: The pressure Poisson equation is solved using Jacobi iteration. Initially, divergence is calculated for each grid cell. Then, pressure values are iteratively adjusted to minimize divergence, enforcing incompressibility. The velocity field is updated by subtracting the pressure gradient.

\item \textbf{Transfer Back to Particles}: Updated grid velocities are interpolated back onto particles.

\item \textbf{Move Particles}: Particles are advected according to their updated velocities. Boundary conditions are enforced by repositioning particles inside the domain and setting boundary normal velocities to zero.
\end{enumerate}

\subsection{PIC/FLIP Implementation}

The PIC/FLIP hybrid method follows a similar pipeline but differs in how particle velocities are updated:

\begin{enumerate}
\item Before applying gravity, the current grid velocity is stored.

\item After solving the pressure, the difference between the new and old grid velocities is computed.

\item Particle velocities are updated using a blend of PIC and FLIP:
\begin{equation}
  \begin{aligned}
  &\text{blended\_velocity} \\
  &= \text{ particle\_velocity} \\
  &+ \text{ flip\_ratio} \cdot (\text{new\_grid\_velocity} - \text{old\_grid\_velocity})
  \end{aligned}
\end{equation}
where a flip\_ratio of 0 corresponds to pure PIC, and values approaching 1 resemble FLIP.

\item Particles are then advected in the same manner as the PIC method, including boundary handling.
\end{enumerate}
% \input{Sections/flip}
\subsection{APIC}
To implement the APIC method, we aimed to simulate incompressible fluid behavior with both stability and visual richness. Compared to PIC or FLIP, APIC introduces an affine velocity field per particle to better capture rotational and shear motion, which helps reduce excessive numerical dissipation and jittering effects.

\subsubsection{Algorithm}

The core steps of the APIC method are summarized in Algorithm~\ref{alg:apic}. This method extends the standard PIC approach by introducing an affine velocity matrix for each particle, which allows capturing local rotational and shear motions more accurately.

\begin{algorithm}[h]
\caption{APIC Particle Update Loop}\label{alg:apic}
\begin{algorithmic}[1]

\For{each particle $p$}
\Comment{Particle to Grid (P2G) }
    \For{each neighboring grid node $g$}
        \State Compute weight $w_{pg}$ and offset $\mathbf{d} = x_g - x_p$
        \State Transfer velocity: $v_g \gets v_g + w_{pg} \cdot (v_p + C_p \cdot \mathbf{d})$
        \State Transfer mass: $m_g \gets m_g + w_{pg}$
    \EndFor
\EndFor

\For{each grid node $g$}
\Comment{Grid Operations(Add Forces)}
    \If{$m_g > 0$}
        \State Normalize: $v_g \gets \frac{v_g}{m_g}$
    \EndIf
    \State Apply gravity: $v_g \gets v_g + \Delta t \cdot g$
    \State Enforce boundary conditions on $v_g$
\EndFor

\For{each particle $p$}
\Comment{Grid to Particle (G2P)}
    \State Initialize: $v_p \gets 0$, $C_p \gets 0$
    \For{each neighboring grid node $g$}
        \State Compute weight $w_{pg}$ and offset $\mathbf{d} = x_g - x_p$
        \State Interpolate velocity: $v_p \gets v_p + w_{pg} \cdot v_g$
        \State Update affine matrix: $C_p \gets C_p + w_{pg} \cdot v_g \otimes \mathbf{d}$
    \EndFor
    \State Update position: $x_p \gets x_p + \Delta t \cdot v_p$
\EndFor
\end{algorithmic}
\end{algorithm}

\subsubsection{Intermediate Results and Diagrams}
We evaluated the performance of the APIC method with different particle counts. Figure~\ref{fig:apic_comparison} shows simulation snapshots at three resolutions—1000, 4000, and 8000 particles—demonstrating how the method handles fluid detail, stability, and distribution over time. Each subfigure compares the final state of the fluid, and highlights how increasing the number of particles leads to smoother, more detailed results.
The APIC method retains coherent motion and prevents clumping or artificial viscosity, which is often observed in simpler schemes like PIC. The particles settle smoothly while preserving rotational features due to the affine velocity transfer.
\begin{figure*}[h]
    \centering
    \begin{subfigure}[b]{0.2\textwidth}
        \includegraphics[width=\textwidth]{figures/apic1000_init.png}
        \caption{1000 initial particles}
    \end{subfigure}
    \hspace{1em}
    \begin{subfigure}[b]{0.2\textwidth}
        \includegraphics[width=\textwidth]{figures/apic1000.png}
        \caption{1000 particles}
    \end{subfigure}
    \hspace{1em}
    \begin{subfigure}[b]{0.2\textwidth}
        \includegraphics[width=\textwidth]{figures/apic4000.png}
        \caption{4000 particles}
    \end{subfigure}
    \hspace{1em}
    \begin{subfigure}[b]{0.2\textwidth}
        \includegraphics[width=\textwidth]{figures/apic8000.png}
        \caption{8000 particles}
    \end{subfigure}
    \caption{Comparison of APIC simulation results with increasing particle counts. 
    (a) shows the initial particle configuration, where all particles are placed in the center of the domain. 
    (b)–(d) show the simulation at the moment particles start to fall under gravity.}
    \label{fig:apic_comparison}
\end{figure*}