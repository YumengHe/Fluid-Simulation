% Introduction
\section{Introduction}

Fluid simulation is a central topic in physics-based graphics and engineering.
Researchers study two broad classes of flow.
Compressible fluids—such as smoke, fire, or drifting snow—change density as they move.
Incompressible fluids—such as water—preserve volume.
Our project narrows its focus to incompressible flow because it underpins many game and film effects.

Scientists have pursued fluid solvers for more than three decades.
Early work in the 1990 s split along two lines. Grid-based methods stored velocity on fixed cells and solved pressure on a lattice.
Particle methods—notably Smoothed Particle Hydrodynamics (SPH)—tracked discrete parcels of mass.
Each line had limits: grids diffused small details, while pure particles struggled with volume loss and boundary handling.

Around 2000, hybrid techniques emerged.
Particle-In-Cell (PIC) used a grid for forces and particles for advection.
FLIP kept the same layout but reduced numerical damping.
Material Point Method (MPM) added elastoplastic behavior for snow-like media.
Affine Particle-In-Cell (APIC) later improved rotational fidelity by carrying local affine velocity.
These methods mix Eulerian and Lagrangian views to balance stability and detail.

Our project builds an interactive framework that implements five representatives: Stable Fluids (grid), SPH (particle), PIC, hybrid PIC/FLIP, and APIC.
We run every solver on the same domain, time step, and boundary conditions. We then measure speed, memory use, and visual artifacts.
The side-by-side view reveals each method's trade-off between diffusion, noise, and stability, and helps artists choose the right tool for a desired effect.


% Contribution
\subsection{Contribution}
This project as the follow contributions.
\begin{itemize}
	\item The codebase supports five fluid solvers behind one interface. Users can swap methods with a single flag.
	\item The viewer renders density, velocity, and vorticity in real time. It uses GLUT for portability.
	\item We fix domain size, time step, and boundary conditions across all tests. This isolates algorithmic differences.
	\item We capture signature phenomena such as diffusion, particle clumping, and energy drift. Screenshots and videos illustrate each effect.
\end{itemize}
